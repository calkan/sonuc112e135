\documentclass[12pt]{article}
\usepackage{lastpage, fancyhdr,color,amsmath,amssymb,amsfonts,amscd, graphicx,latexsym}
\usepackage{eurosym}
\usepackage[all]{xy}
\pagestyle{empty}
\usepackage[turkish]{babel}
\usepackage[utf8]{inputenc}
\usepackage[T1]{fontenc}
\setlength\voffset{-1in}
\setlength\hoffset{-1in}
\setlength\topmargin{3cm}
\setlength\oddsidemargin{2.5cm}
\setlength\textheight{8in}
\setlength\textwidth{6.51in}
\setlength\footskip{1in}
\setlength\headheight{12pt}
\setlength\headsep{0.06in}
\usepackage{tikz}
%plus4mm minus3mm}
\usepackage{caption, subcaption, amsfonts}
\usepackage{setspace}
\usepackage{epstopdf}
\usepackage{relsize}
\usepackage{pgf, float}
\usepackage{hyperref}
\newcommand{\gal}{Gal({\mathbb Q})}
\newcommand{\N}{\mathbb N}
\newcommand{\F}{\mathcal F}
\renewcommand{\H}{\mathcal H}
\newcommand{\Q}{\mathbb Q}
\newcommand{\Z}{\mathbb Z}
\newcommand{\R}{\mathbb R}
\newcommand{\C}{\mathbb C}
\newcommand{\p}{\mathbb P}
\newcommand{\B}{\mathbb B}
\def\modorb{\mbox{$\otimes\hspace{-1.5mm}-\!\!\!-\!\!\!-\hspace{-1.5mm}\circledast$}}
\hyphenation{or-bi-fold}
%\input{defs.tex}

%\usepackage[usenames,dvipsnames]{color}
%\usepackage{epsfig}


\definecolor{light-gray}{gray}{0.55}
\newcommand\Note[1]{\textcolor{red}{{#1}}}
\begin{document}
\pagestyle{fancy}
   \lhead{}\rhead{}
   \lfoot{\textcolor{light-gray}{\small Gal-Act Proje Sonuç Raporu}}
        \rfoot{\textcolor{light-gray}{\small Sayfa {\thepage}}}
        \cfoot{}
  \renewcommand{\headrulewidth}{0pt}
%  \renewcommand{\footrulewidth}{0.4pt}
%\def\HG{{\bf HG-GalAct }}

\label{coverpage}



\newpage
\phantom{22}
\vspace{-3cm}

% % % {\selectlanguage{turkish}\bfseries\color{red} \small Başvuru Formunun ``Bütçe ve Gerekçesi (13. Madde)'' dışındaki bölümleri toplamda Arial, 9 yazı tipinde 20 sayfayı geçmemelidir!}
% % % 
% % % \begin{center}
% % % 	\selectlanguage{turkish}\bfseries \footnotesize Kariyer proje önerisi değerlendirme formuna \\ \url{http://www.tubitak.gov.tr/tubitak_content_files/ARDEB/destek_prog/danisman_panelist/3501_DA_Panelist_Proje_Onerisi_Degerlendirme_Formu.doc} \\
% % % 	adresinden ulaşabilirsiniz.
% % % \end{center}


\begin{center}
\includegraphics[keepaspectratio]{tubitak.png}

\bigskip
\bigskip


\bigskip
{\fontsize{15}{10}\selectfont 
\bigskip


\bigskip
\medskip
{ \textbf{\Huge Birden Fazla Veri Kaynağı Kullanabilen Yeni Genom Birleştirme Algoritmalarının Tasarımı Ve Uygulanması \\}}}


\bigskip
\medskip
{ \textbf{ PROJE SONUÇ RAPORU}}
\bigskip





\end{center}


\bigskip

\bigskip


\thispagestyle{empty}


\begin{center}
\medskip
{\LARGE \textbf{Program Kodu:} 1001}

\bigskip
{\LARGE \textbf{Proje No:} 112E135}

\bigskip
{\LARGE Proje Yürütücüsü:\\
\textbf{Can Alkan}}

\end{center}



\bigskip


\bigskip
\noindent
{\Large
\noindent
\underline{\bf Bursiyerler:}

\noindent
Shatlyk Ashyralyyev

\noindent
Fatma Kahveci (Balcı)

\noindent
Elif Dal


}



\bigskip





\begin{center}
{\Large EYLÜL 2015

\vspace{1mm}
ANKARA}
\end{center}


\linespread{1.5}

\newpage\setlength{\parskip}{3mm} 
\onehalfspacing
\bigskip
\setcounter{page}{1}
\begin{center}
{\LARGE \bf ÖNSÖZ}
\end{center}
\addcontentsline{toc}{section}{ÖNSÖZ}


Proje süresince proje ekibinin şu yayınları hazırlanmıştır: 

\begin{itemize}
\item Early post-zygotic mutations contribute to de novo variation in a healthy monozygotic twin pair. G.M. Dal, B. Ergüner, M. S. Sağıroğlu, B. Yüksel, O. E. Onat, {\bf C. Alkan}, T. Özçelik. Joural of Medical Genetics, 51:455-459, 2014.
\item Whole genome sequencing of Turkish genomes reveals functional private alleles and impact of genetic interactions with Europe, Asia and Africa. {\bf C. Alkan}, P.  Kavak, M. Somel, O. Gokcumen, 
  S. Uğurlu, C. Saygı, {\bf E. Dal}, K. Buğra-Bilge,  T. Güngör, S. C. Sahinalp, N. Özören, C. Bekpen. BMC Genomics, 15(1):963, 2014.
\end{itemize}

Proje süresince proje ekibinin şu bildiriler sunulmuştur:
 
\begin{itemize}
\item A hypergraph model for hybrid genome assembly. {\bf S. Ashyralyyev}, C. Firtina, C. Aykanat, {\bf C. Alkan}. Bertinoro Computational Biology Meeting, 14-18 May 2015, Bertinoro, Italy.

\end{itemize}

Proje kapsamında desteklenen aşağıdaki bursiyerler,  lisansüstü eğitimlerini 
proje yürütücüsünün danışmanlığında
yapmıştır.
 
\begin{itemize}
\item Fatma Kahveci (Balcı), Bilkent Üniversitesi Bilgisayar Mühendisliği, Yüksek Lisans, Ağustos 2014.
\item Elif Dal, Fatma Kahveci (Balcı), Bilkent Üniversitesi Bilgisayar Mühendisliği, Yüksek Lisans, Aralık 2014.
\end{itemize}


\bigskip
\hfill Can Alkan

\hfill Ankara, Eylül 2015
\newpage

\setlength{\parskip}{1mm} 

\tableofcontents


 



\newpage \setlength{\parskip}{3mm}
\phantom{ss}
\vspace{-2.5cm}

\begin{center}
{\bf \Large ÖZET}: 
\end{center}
\addcontentsline{toc}{section}{ÖZET}
\noindent
Özet
Yeni nesil dizileme (YND) teknolojilerinin uygulanmasi genomiks alanını kökten değiştirmektedir. Farklı türlerin genomlarini incelemede ve gerek normal gerekse hastalığa yol açan insanlardaki genetik farklılıkların incelenmesinde önceden beklenmeyen derecede çözünürlük sağlamaktadır. Her ne kadar YND verilerinin analizi için çok önemli gelişmeler olmuşsa da, halen YND yöntemlerinin tüm gücünden faydalanılmasının önünde engeller vardır.
Önceden hayal dahil edilemeyecek hızda verileri günümüzde üretebiliyor olmamıza rağmen, bu verilerin analizleri çok daha yavaş hızda olmaktadır. Çünkü 1) emsalsiz miktarlardaki veriler bilişimsel altyapılarda hem verilerin saklanması hem de işlenmesi açısından sorunlar doğurmaktadır; 2) YND platformları tarafından üretilen dizi parçaları genelde yüksek oranda hata içermektedir ve bu parçalar çok kısadır; 3) hem halihazırda kullanılmakta olan algoritmalar hem de YND platformları genomun bazı farklı yapıdaki bölgelerinde düşük performanslı çalışmaktadırlar. İşte bu nedenlerden ötürü dizileme verilerinde var olan bilgiler tamamen kullanılamamaktadır. Bu çok miktardaki verilerin daha iyi işlenmesi ve YND yöntemlerinin gerçek gücünün ortaya çıkarılması için bilgisayar bilimleri ve genomiks arasında bir işbirliğinin kurulması gerekmektedir.
Genom dizileme maliyetinin büyük ölçüde düşmesi sayesinde farklı organizmalar arasındaki genomik çeşitliliğin, organizmal biyolojinin ve genom evriminin daha iyi anlaşılması için binlerce farklı türün genomlarının dizilenmesine yönelik büyük bir ilgi vardır. Son bir kaç yılda bir çok genom, örneğin pirinç, üzüm, buğday, patates, mısır ve salatalık gibi bitkiler; panda, hindi, goril, orangutan, bonobo, keseli sıçan (opossum) ve fil gibi hayvanların genomlari dizilenmistir. Yakın zamanda Genome 10K gibi çok iddialı projeler başlamış ve 10.000 omurgalı hayvanın tüm genom dizilenmesinin yapılması amaçlanmıştır. Ancak YDS platformlarıyla üretilen verilerin yukarıda bahsedilen sınırları farklı türlerin referans genomlarının ortaya çıkarılmasını amaçlayan yeni dizileme çalışmalarını da (de novo sequencing) olumsuz etkilemektedir. Bunun başlıca nedenleri çoğu türün genomlarında aynı ya da benzer DNA dizilerinin genomun farklı yerlerinde tekrarlanması, YND verilerindeki dizi parçacıklarının (sequence read) kısa olması ve yüksek oranda hataların bulunmasıdır. Bu nedenle tüm genom dizilerinin çıkarılması sırasındaki genom birleştirmenin (genome assembly) doğruluğunun arttırılması için halen çözülmesi gereken problemler bulunmaktadır. Dizilenmiş genomların doğruluğu yetersiz derecede olursa, bu genomların incelenmesinden çıkarılacak biyolojik sonuçlar da hatalı olacaktır.
Bu projede birden fazla veri türünü birlikte kullanabilecek algoritmaların tasarlanmasını ve uygulanmasını öneriyoruz. Farklı şekillerde üretilmiş genom dizileme verilerinin gösterdiği farklı özellikleri aynı anda kullanarak ortaya çıkarılacak referans genomlarının kalitesi arttırılacaktır. Böylece bu proje dahilinde üreteceğimiz gelişmiş algoritmalar yeni dizilenmiş genomları daha iyi birleştirmemizi sağlayacak ve genom biyolojilerini daha iyi anlamamıza yardımcı olacaktır.

\newpage
\phantom{ss}
\vspace{-2.5cm}


\begin{center}
{\bf \Large ABSTRACT}: 
\end{center}
\addcontentsline{toc}{section}{ABSTRACT}
\noindent
The application of high throughput sequencing (HTS) technologies are revolutionizing the field of genomics, providing unprecedented resolution to study genomes of different species, and normal and disease causing human genetic variation. Although significant advances have been made to analyze HTS data, there are still several hurdles in fully utilizing the power of HTS. 
Although we can now generate data at a rate previously unimaginable, the analysis of the data is proceeding at a slower pace because: 1) unprecedented amounts of data introduce challenges in computational infrastructure in terms of both storage and processing power; 2) reads are often associated with high sequence errors and shorter read length; and 3) currently available algorithms to analyze HTS data and the HTS data themselves show different biases against different regions of the genome. Due to these problems, the information available in the sequencing datasets is not completely mined. There is a need to forge an alliance between computer science and genomics to devise better methods to use the massive amount of sequence data to unleash the full power of HTS methodologies.
Thanks to the substantially reduced cost of genome sequencing, there is now great interest in sequencing the genomes of thousands of species to better understand the genomic diversity across different organisms, organismal biology and genome evolution. In the last few years many genomes are sequenced: plants such as rice, grape, wheat, potato, corn, cucumber; and animals such as the giant panda, turkey, gorilla, orangutan, bonobo, opossum, elephant, etc. Recently more ambitious projects like the Genome 10K Consortium are started to sequence the genomes of 10.000 vertebrate species. However, the aforementioned limitations of the HTS technologies also affected de novo sequencing studies that aim to construct the reference genomes of various species. This is mainly due to the repetitive structure of the genomes of most species, the short sequence reads generated by current platforms, and the increased error rate. Thus there are still problems to solve to increase the accuracy of the assembled genomes; otherwise any biological conclusions derived from non-accurate genome assemblies would be incorrect.
Reasoning from the previous observations and empirical evidence that all current HTS platforms show different strengths and biases, we propose to devise novel genome assembly algorithms that use data from multiple sources, including, when available, data derived from laboratory experiments to better assemble the genomes of new species. We will test our algorithms with 1) a set of bacterial artificial chromosomes (BACs) generated from a hydatidiform mole resource that were sequenced using both the Illumina and Pacific Biosciences platforms, and test the assembly accuracy by comparing with high quality assemblies of the same resource using capillary sequence data; 2) whole genome shotgun sequence libraries generated from a haploid genome (hydatidiform mole) and sequenced using 454/Roche and Illumina platforms, several BAC end sequences from the same library sequenced using capillary sequencing, and physical fingerprinting data. The basepair calling accuracy of the Illumina platform coupled with longer matepairs from the 454/Roche, long sequences from Pacific Biosciences, long “jumps” from BAC end sequencing, and the physical ordering of the BACs from the fingerprint data will be used in harmony to improve the genome assembly. In long term, we will also incorporate methodologies that utilize data from upcoming nanotechnology-based sequencing platforms such as the Oxford Nanopore Technologies. Enhanced algorithms that can better assemble genomes will improve our understanding of the biology of genomes.

\vspace{-.5cm}

\newpage
\begin{center}
{\bf \Large 1. GİRİŞ}
\end{center}
\addcontentsline{toc}{section}{1. GİRİŞ}

\bigskip
\noindent
{\bf \large 1.1 }
\addcontentsline{toc}{subsection}{1.1 }

\noindent

\begin{center}
{\bf \Large 2. LİTERATÜR ÖZETİ}: 
\end{center}
\addcontentsline{toc}{section}{2. LİTERATÜR ÖZETİ}


\begin{center}
{\bf \Large 3. BULGULAR}: 
\end{center}
\addcontentsline{toc}{section}{3. BULGULAR}
\noindent

\noindent
{\bf \large 3.1. }
\addcontentsline{toc}{subsection}{3.1 }

\begin{center}
{\bf \Large 4. TARTIŞMA/SONUÇ}: 
\end{center}
\addcontentsline{toc}{section}{4. TARTIŞMA/SONUÇ}

\noindent
Yürürlükte olduğu üç sene boyunca, **** projesinin ele aldığı soruların bir kısmında kayda değer bir ilerleme kaydetmiştir. 

\noindent
{\bf \Large 4.1. Öneriler}
\addcontentsline{toc}{section}{4.1. Öneriler}

{\small 
%BBBBBBBBBBBBBBBBBBBB
\begin{thebibliography}{99}\label{biblio}
%*************************

%\setcounter{enumiv}{22}


}
\end{thebibliography}




\label{endsectionb1}
\end{document}
